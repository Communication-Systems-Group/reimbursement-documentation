\chapter*{Abstract}
\addcontentsline{toc}{chapter}{Abstract}

\selectlanguage{german}

Der Spesenrückvergütungsprozess der Universität Zürich ist für deren Mitarbeitende  gegenwärtig ein unhandliches und zeitaufwändiges Unterfangen. Zudem wird durch Medienbrüche unnötig viel Papier verschwendet. Zur Zeit werden Spesenformulare in MS Excel erfasst, danach ausgedruckt und zusammen mit den Spesenbelegen händisch von Prozessinstanz zu Prozessinstanz weitergegeben. Dadurch, dass die Dokumente in Papierform weitergegeben werden, kann die Validierung und das Unterzeichnen nur am Arbeitsplatz erfolgen. Dies macht den Prozess im Zeitalter der individuellen Arbeitszeitgestaltung und des Home-Office ineffizient und fehleranfällig. In den vergangenen Jahren ist durch die zunehmende Digitalisierung der Arbeitswelt der Wunsch nach einer einheitlich digitalen Lösung entstanden. In dieser Arbeit wird in einem ersten Schritt ein Teilprozess, nämlich derjenige der Spesenerfassung und -validierung, in einem Prozessmodell visualisiert und danach validiert. Dies in enger Zusammenarbeit mit dem Lehrstuhl von Prof. Dr. Burkhard Stiller am Institut für Informatik. In einem zweiten Schritt und auf der Basis des validierten Prozessmodelles wird eine Softwarelösung spezifiziert und implementiert. Ziel der Arbeit ist die Reduktion des zeitlichen Aufwandes sowie des Papierverbrauchs.


\selectlanguage{english}

The reimbursement-process at University of Zurich is currently cumbersome, wasteful and therefore inefficient. Expense receipts need to be captured in an MS Excel form, then printed and assembled to a document before being passed by hand to the next instance in the process in order to be accepted and approved. As a consequence, the process is time-consuming and error prone. Besides that, new work-patterns like individual working-habits and home-office increased the wish for a digital and location independent way of expense-reimbursement. In this work, in a first step the official and by the Finance Department of the University of Zurich enforced reimbursement process is visualized in a process model. This is done at the Department of Informatics and in close collaboration with the chair of Prof. Dr. Burkhard Stiller. In a second step, a software artifact was specified and finally implemented. The overall goal of the work is to reduce the time employees use to complete the reimbursement-creation process and, additionally, to reduce paper-waste.
