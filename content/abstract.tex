\chapter*{Abstract}
\addcontentsline{toc}{chapter}{Abstract}

\selectlanguage{german}

Der Spesenrückvergütungsprozess der Universität Zürich ist für deren Mitarbeitende gegenwärtig ein unhandlicher und zeitaufwändiger Prozess. Zudem wird durch Medienbrüche unnötig viel Papier verschwendet. Zur Zeit werden Spesen in MS Excel erfasst, danach ausgedruckt und zusammen mit den Spesenbelegen händisch von Prozessinstanz zu Prozessinstanz weitergegeben. Dadurch, dass die Dokumente in Papierform weitergegeben werden, kann die Validierung und das Unterzeichnen nur am Arbeitsplatz erfolgen. Dies macht den Prozess im Zeitalter des Home Office und internationalen Konferenzen ineffizient und Fehleranfällig. In den vergangenen Jahren ist durch die zunehmende Digitalisierung der Arbeitswelt der Wunsch nach einer einheitlichen und digitalisierten Lösung entstanden. Das Ziel dieser Arbeit ist es in einem ersten Schritt aufzuzeigen, wie der gegenwärtige und von der Finanzabteilung der Universität Zürich vorgeschriebene Prozess digitalisiert und in einer Softwarelösung implementiert werden kann. Im zweiten Schritt wird diese Softwarelösung spezifiziert, implementiert und in Zusammenarbeit mit dem Lehrstuhl von Prof. Dr. Stiller am Institut für Informatik getestet. Dabei besteht das grundlegende Ziel in der Reduktion des zetlichen Aufwandes für alle Beteiligten sowie in der Minimierung des Papierverbrauches.

\selectlanguage{english}

The reimbursement-process at the Department of Informatics and the University of Zurich is cumbersome, wasteful and therefore inefficient. Expense receipts need to be captured in an MS Excel form, then printed and assembled to a document before being passed by hand to the next instance in the process for acceptance and approval. As a consequence, the process is time-consuming and error prone. Besides that, new work-patterns like home-office and the attendance of the researchers in international conferences increased the wish for a digital and location independent way of expense-reimbursement. This work aims at showing in a first step how the official and by the Finance Department of the University of Zurich enforced reimbursement process can be digitalized and how this process can be implemented in a software solution. In a second step, a software artifact is designed, implemented and tested at the Department of Informatics in collaboration with the chair of Prof. Dr. Stiller. The overall goal of the work is to reduce the time employees use to complete the charge-back process and moreover to reduce paper-waste.
