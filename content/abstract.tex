\chapter*{Abstract}
\addcontentsline{toc}{chapter}{Abstract}

\selectlanguage{german}

Der Spesenrückvergütung der Universität Zürich ist für deren Mitarbeitende gegenwärtig ein unhandlicher und zeitaufwändiger Prozess. Zudem wird durch Medienbrüche unnötig viel Papier verschwendet. Zur Zeit werden Spesen in MS Excel erfasst, danach ausgedruckt und zusammen mit den Spesenbelegen händisch von Prozessinstanz zu Prozessinstanz weitergegeben. Dadurch, dass die Dokumente in Papierform weitergegeben werden, kann die Validierung und das Unterzeichnen nur am Arbeitsplatz erfolgen. Dies macht den Prozess im Zeitalter der individuellen Arbeitszeitgestaltung und des Home-Office ineffizient und Fehleranfällig. In den vergangenen Jahren ist durch die zunehmende Digitalisierung der Arbeitswelt der Wunsch nach einer einheitlich digitalen Lösung entstanden. In dieser Arbeit wird in einem ersten Schritt der Prozess der Spesenerfassung in einem Prozessmodell visualisiert und validiert. Dies wurde in enger Zusammenarbeit mit dem Lehrstuhl von Prof. Dr. Stiller am Institut für Informatik durchgeführt. Danach wird in einem zweiten Schritt auf der Basis des validierten Prozessmodelles eine Softwarelösung spezifiziert und implementiert. Ziel der Arbeit ist die Reduktion des zeitlichen Aufwandes sowie des Papierverbrauchs.


\selectlanguage{english}

The reimbursement-process at the Department of Informatics and the University of Zurich is cumbersome, wasteful and therefore inefficient. Expense receipts need to be captured in an MS Excel form, then printed and assembled to a document before being passed by hand to the next instance in the process for acceptance and approval. As a consequence, the process is time-consuming and error prone. Besides that, new work-patterns like individual work-habits and home-office increased the wish for a digital and location independent way of expense-reimbursement. In this work, in a first step the official and by the Finance Department of the University of Zurich enforced reimbursement process is visualized in a process model. This was done at the Department of Informatics and in close collaboration with the chair of Prof. Dr. Stiller. In a second step, a software artifact was specified and finally implemented. The overall goal of the work is to reduce the time employees use to complete the charge-back process and moreover to reduce paper-waste.
