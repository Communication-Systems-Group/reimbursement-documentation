\chapter{Conclusions}
 
\section{Security Information}
The currently implemented electronic signing is, in contrast to the digital signing, not bullet proof. Although the reimbursement-tool prevents any user (using our front-end) to edit a document that has been signed electronically, any user can manipulate the generated pdf or printed expense document before submitting it to the Finance Administration of the University in paper form. This is a flaw that could be mitigated by shifting the responsibility for sending the final document to IFI's finance administration.
 
 
\section{Summary}
The reimbursement process at the Department of Informatics (IFI) at the University of Zurich was based on a MS-Office Excel form that has been used to capture the expenses and their receipts and pass them physically through the entire reimbursement process. For the participants this was impractical and time consuming: they had to pass the digital Excel file and the physical receipts to the next member in the process. Moreover caused any change a reprinting of at least the expense overview page of the document which resulted in a vast waste of paper.\par

In order to make at least the reimbursement-creation-process as smooth as possible, we developed a web application that supports Expense Creators, Managers and the Finance Administration personnel in their respective tasks in the process. The web application reduces the overall efforts needed to create, validate and sign expenses and likewise helps all participants to focus on their work which results in an overall raised productivity and, what is also important, in lower operational costs (paper). Besides that, the Finance Administration of the IFI department gets an analytical tool to gain a better knowledge of the state-distribution in the reimbursement-process. The integrated search engine allows them to search, retrieve and reset expenses in only a few seconds. \par

The reimbursement-tool in its current state fulfills all requirements: Expenses can be created, modified, accepted, rejected and passed along the specified process. They can be printed as a pdf and the pdf document obeys the specification given by the Finance Administration of the University of Zurich. \par

Moreover users can conveniently login to the system by using their IFI LDAP credentials and the integrated currency-conversion-service and the camera-device integration reduces the efforts for the adding of expense receipts to a minimum. The creation is not only faster, but also less error prone.\par

As an extra, the reimbursement-tool provides the "Guest View". The guest view is a view especially designed and implemented to allow an anonymous user (e.g the Finance Administration of the University) to view a dedicated expense with related expense items and all other information. This is a "Goodie"-Feature which should be used to convince the Finance Administration of the UZH to re-think the current absurdity and inefficiency in their end-to-end process.\par

However, during the project, several other possibilities for future improvements have emerged. These are summarized in the next chapter.

\section{Future Work}
\label{sec:future-work}

The current implementation of the reimbursement-tool covers the basic features to run the entire reimbursement-creation-process digitally without any media disruption. However, there are still some possibilities for future improvements.

\subsection{Technical Improvements}
\subsubsection{New Excel Form Version}
During the development of the reimbursement-tool, a new specification of the MS-Excel form has been announced. This form will be introduced in mid 2016 and the current PDF-generation has to be adapted in time. Obviously these new requirements need to be implemented in a future step.

\subsubsection{Spring Profiles}
Spring Profiles would allow a convenient and nice way to activate and deactivate single extensions like the e-mail service, swagger-ui or the LDAP synchronisation.

\subsubsection{Mitigation of Performance-Bottlenecks}
During load-testing of our application, we spotted potential memory bottlenecks.\par 

\textbf{Image conversion:} If users upload large images (> 8mb), the conversion of the image-formats to the PDF-format consumes quite a lot of the available heap space. \textit{Work-around:} there exist libraries that allow the pdf creation already on the client.\par 
 	
\textbf{Large Amount of expenses:} If a lot of expenses (> 100) have to be serialized and sent in a single request, the serialization can lead to an OutOfMemoryError and is visible as a HTTP 500 error code in the frontend together with a stacktrace. Since Spring only catches exceptions but no errors, this error is not transformed to our nice default error message. \textit{Workaround: } To mitigate this large number of expenses in a single request, pagination could be implemented. The client requests only a defined number of expenses based on some filter criteria. All other expenses are loaded only on demand and in chunks of a certain size.


\subsection{Process Improvements}
\label{sec:process-improvements}
\subsubsection{Process integration to UZH}
Currently the expenses will be printed and delivered to the Finance Administration of the University of Zurich in paper format. In turn, they will be scanned to be digitalized and archived at the Finance Administration of the University of Zurich. If this media disruption could be suspended, the overall process efficiency would increase. The system is ready to integrate the last instance in the process - the Finance Administration of the University of Zurich. The system provides a guest-view to point out the value of the tool.\par

\subsubsection{Further Automation}
If we assume that the Finance Administration of UZH allows a pure digital form of data-exchange, then further improvements are possible: Since the reimbursement-tool provides already an e-mail service, it would make sense to automate the e-mail sending. Instead of printing the document, the e-mail service could send the document to a defined e-mail address. Another option would be to provide a public API to the Finance Department, to enable a programmatical data exchange.\par 

\subsection{Feature Improvements}
\subsubsection{Employee bank account data}
In conversations with employees at the IFI, we often heard complaints about wrong personal data. Our reimbursement-tool could also be extend to provide more information about the Expense Creators: this could be the account data, address and other personal information. Storing this information in the reimbursement system offers the benefit that users can update their information at any time and the Finance Administration of the UZH is always up-to-date regarding account-number changes etc.

\subsubsection{Advanced Statistics for Finance Administrators } 
We implemented some basic overviews and visualizations of the current expenses. For the finance administrators it would probably be cool to have more sophisticated tools and the possibility to export the data.

\subsubsection{Real Archiving for Sysadmins} 
Currently the reimbursement system allows to archive an expense. This allows to separate active and archived expenses in different requests. The archived expenses remain in the same database as also the active ones. Currently, there is no "easy" way to move archived expenses from one database to another.
