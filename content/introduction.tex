\chapter{Introduction}

\section{Motivation}

The current reimbursement process at the University of Zurich is cumbersome and contains several media disruptions: Excel forms have to be filled in, printed, reviewed, adapted, printed and reviewed again. Numerous iterations lead in consequence to a vast waste of paper and time. Furthermore, travel expenses as well as other expenses that are already present in a digital form need to be printed and attached to the other paper documents before being passed manually to the next instance in the process. As last step, the financial administration of the IFI reviews and signs the final form and sends it by postal-mail to the finance departement of the University of Zurich in order to trigger the actual cash-flow back to the employee \cite{ifi}. There, at the finance departement of UZH, the post-mail is scanned and saved as pdf-document.

\section{Description of Work}

The aim of this master-project is to design and implement a web-based portal and its back-end functionality to support the general reimbursement process with a centralized and database-driven approach. The work consists of the design of an information model, the relational database model for storing reimbursement data, and the respective new travel reimbursement form. Furthermore, the development of the web-based portal and applications for displaying and inputting reimbursementinformation, the supervisor’s approval site, the respective financial department’s portal, and the technical administrative web site are needed. Finally, the entire web portal and application, which shall maintain in the future the entirety of all IFI-based travel reimbursement information, shall also be tested in CSG-internal approach.


\section{Project Scope}

In this master-project the implementation of the currently used features of the reimbursement process at the Department of Informatics at the University of Zurich is conducted. This consists the complete process of creating, reviewing, rejecting, accepting and printing of expense elements. As well as e-mail notifications of the participants if their task state change. Furthermore the printed Expense-Pdf needs to have an identical design according to the existing Excel-Pdf.\newline 
To support a high user satisfaction, users need to be able to login with existing credentials. This is realised by enabling the users to login with their existing IFI LDAP user account. Besides that the overall GUI structure and design needs to be appealing and responsive.