\chapter{Introduction}

\section{Motivation}

The current reimbursement process at the University of Zurich (UZH) is cumbersome and contains several media disruptions: Excel forms have to be filled in, printed, reviewed, adapted, printed and reviewed again. Numerous iterations lead in consequence to a vast waste of time and paper. Furthermore, travel expenses as well as other expenses that are already present in a digital form need to be printed and attached to the other paper documents before being passed manually to the next instance in the process. As last step, the financial administration of the Department, in our case the Department of Informatics (IFI), reviews and signs the final form and sends it by internal postal-mail to the finance administration of the University of Zurich in order to trigger the actual cash-flow back to the employees \cite{ifi}. There, at the finance administration of UZH, the post-mail is scanned and saved as pdf-document, whereas the paper document is archived.

\section{Description of Work}

The aim of this master-project is to design and implement a web-based portal and its back-end functionality to support the general reimbursement process with a centralized and database-driven approach. The work consists of the design of an information model, the relational database model for storing reimbursement data, and the respective reimbursement form. Furthermore, the development of the web-based portal and applications for displaying and inputting reimbursement information, the supervisor’s approval site, the respective financial administration’s portal, and the technical administrative web site are needed. Finally, the entire web application, which shall maintain in the future the entirety of all IFI-based travel reimbursement information, shall also be tested in a CSG-internal approach.


\section{Project Scope}

In this master-project the implementation of the currently used features in the reimbursement-creation process at the Department of Informatics is conducted. This sub-process of the end-to-end reimbursement process is restricted to creating, reviewing, rejecting, validating, signing and printing of expenses and related expense receipts. Furthermore the printed Expense-Pdf needs to have an identical design corresponding to the existing MS Excel form. To inform the users about expense state changes or other issues that require their attention, e-mail notifications have to be implemented. To support a high user satisfaction, users need to be able to login with existing credentials. This is realised by enabling the users to login with their existing IFI LDAP user account. The overall GUI structure and design should be appealing and responsive.