\chapter{Introduction}

\section{Motivation}

The actual reimbursement process at the University of Zurich is cumbersome and contains several media disruptions. Excel documents will be created, printed, reviewed, adapted, printed and reviewed again. Numeral iterations lead to waste of paper, time and energy. Furthermore travel expenses as well as expenses need to be printed, attached, scanned and reviewed by the financial administration of each IFI \cite{ifi} department. To reduce the efforts taken for reimbursement tasks, especially on creation and review actions during the process for all participants, a new Web-based portal will be developed.

\section{Description of Work}

The aim of this master-project is to design and implement a Web-based portal and its back- end functionality to operate the travel reimbursement process with a centralized and database-driven approach. The work consists of the design of a information model, the relational database model for storing reimbursement data, and the respective new travel reimbursement form. Furthermore, the development of the web portal and applications for displaying and inputting reimbursementinformation, the supervisor’s approval site, the respective financial department’s portal, and the technical administrative Web site are needed. Finally, the entire web portal and application, which shall maintain in the future the entirety of all IFI-based travel reimbursement information, shall also be tested in CSG-internal approach.


\section{Project Scope}

In this master-project the implementation of the currently used features of the reimbursement process at the IFI \cite{ifi} department at the University of Zurich. This consists the complete process of creating, reviewing, rejecting, accepting and printing of Expense elements. As well as E-mail notifications of the participants if their task states change. Furthermore the printed Expense-Pdf needs to have an identical appearance and functionality to the existing Excel-Pdf. The users need to login with their existing UZH user account. This can be achieved by connecting the system to the existing LDAP server of the IFI department to fetch user data.  