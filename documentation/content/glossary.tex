\chapter*{Glossary}
\addcontentsline{toc}{chapter}{Glossary}
\markboth{GLOSSARY}{}


\begin{description}
    \item[API] is an abbreviation for Application Programming Interface in computer programming. It can be used to access services of external software using an interface.
    \item[Build process] Is a process to generate a software artifact that contains the complete software code.  
    \item[Client] is the physical hardware entity that a normal user uses i.e. Personal Computer.
    \item[CSS] is an abbreviation for Cascading Style Sheets. It is used to format and style plain HTML code. 
    \item[Dependency management] is used to ensure that the compatibility between various packages is supported.
    \item[Expense] is, in the context of this document an entity that consists out of 1 - 15 receipts. It is the digital counterpart to the physical expense.
    \item[Finance Administration of Zurich] is the head of all finance administration departments of the University of Zurich.
    \item[GUI] is an abbreviation for Graphical User Interface. 
    \item[HTTP] is an abbreviation for Hypertext Transfer Protocol. The protocol is used to transmit HTML data between the client and the server.
    \item[HTML] is an abbreviation for Hypertext Markup Language. It is used to structure digital documents with content like text, images and hyperlinks. 
    \item[IFI] is an abbreviation for the Department of Informatics at the University of Zurich.
    \item[Java] is a widely used object-oriented programming language.
    \item[LDAP] is an abbreviation for Lightweight Directory Access Protocol. It's a network protocol that is used to access and manage distributed directory access and user authorization information. 
    \item[Media disruption] occurs, when a digital document is printed and transferred back to a digital format by hand.
    \item[MVC] is an abbreviation for Model View Controller. It is a widely used software pattern.
    \item[Pdf] is an abbreviation for Portable Document Format. It is a widely used file format, to present documents in most operating system in a fixed layout.
    \item[Receipt] is, in the context of this document the digital counterpart of a physical receipt.
    \item[Reimbursement-tool] is the developed prototype.
    \item[RESTful] is an abbreviation for representational state transfer in computing. Its a definition, on how to access web-based resources in the right way.
    \item[SQL] is an abbreviation for Structured Query Language. It is used to execute database calls to retrieve data in relational databases.
    \item[UI] is an abbreviation for User Interface. 
    \item[UZH] is an abbreviation for University of Zurich.
    \item[XML] is an abbreviation for Extensible Markup Language. It defines a set of rules that are machine and human readable.
    \item[XSL] is an abbreviation for Extensible Stylesheet Language. It is used to transform and render XML documents.
    \item[XSLT] is an abbreviation for Extensible Stylesheet Language Transformations. It is a special language used to transform XML documents.
\end{description}
